\documentclass[12pt,letterpaper]{article}
\usepackage[utf8]{inputenc}
\usepackage[spanish]{babel}
\usepackage{amsmath}
\usepackage{amsfonts}
\usepackage{amssymb}
\usepackage{makeidx}
\usepackage{graphicx}
\usepackage[left=2cm,right=2cm,top=2cm,bottom=2cm]{geometry}
\author{john negrete}
\title {primer avance}

\begin{document} 

\title {primer avance}

\section*{                  ...(PRIMER AVANCE)... }
\section*{John Negrete..........(7-B)}
\section*{Benjamin Enciso..........(ing.Mecatronica)}
\section*{Leonardo Contreras.......(cinematica de robots)}
\section*{Martin Barajas......(Moran Garabito)}
\includegraphics[scale=2]{up1.jpg} 
\section*{INTRODUCCION}
Un robot puede ser definido como una máquina que efectúa un número de trabajos, mediante la programación previa. Una peculiaridad de los robots es su estructura de un brazo mecánico y otra su adaptabilidad a diferentes herramientas.
Por siglos el ser humano ha construido máquinas que imiten las partes del cuerpo humano. Los antiguos egipcios unieron brazos mecánicos a las estatuas de sus dioses. Estos brazos fueron operados por sacerdotes, quienes clamaban que el movimiento de estos era inspiración de sus dioses. Los griegos construyeron estatuas que operaban con sistemas hidráulicas, los cuales se utilizaban para fascinar a los adoradores de los templos.
El uso de sistemas robóticos en la industria, para cumplir funciones que requieren extrema precisión ha ido en ascenso en las últimas décadas como también en el uso personal y familiar.
El desarrollo de estos sistemas se ha enfocado en mejorar ciertos aspectos como resistencia para trabajar en diferentes condiciones, precisión con la que se realizan movimientos, multifuncionalidad (manipulación, corte, perforación, etc.), adaptabilidad en diferentes entornos de trabajo.
Por lo tanto, dados todas estas utilidades, el diseño propio y construcción de prototipos de brazo robótico para manipulación, corte láser o escaneo tengan un costo accesible tanto para la industria como para la educación, es un buen tema a considerar como proyectos de desarrollo, por estudiantes de ingeniería mecatrónica.
El desarrollo en la tecnología, donde se incluyen las computadoras, los actuadores de control retroalimentados, transmisión de potencia a través de engranes, y la tecnología en sensores han contribuido a flexibilizar los mecanismos autómatas para desempeñar tareas dentro de la industria. La investigación en inteligencia artificial desarrolló maneras de emular el procesamiento de información humana con computadoras electrónicas.
\section*{Justificacion}
El hecho de soldar de manera automatizada con microalambre va a resolver problemáticas importantes para el dueño, tales como la reducción de tiempo muerto entre soldar dos materiales y ensamblar piezas o esmerilarlos para quitar la escoria y dejar la soldadura limpia, esto podrá ser un proceso semiautomático o automático que sea menos dependiente de la habilidad de operador.
Un factor importante en el proyecto es usar MIC/MAC ya que es intrínsecamente más productiva que la soldadura MMA en la que se hace una parada cada vez que se consume el electrodo además de hacer poca formación de gases contaminantes y tóxicos.
Las principales bondades de este proceso son la alta productividad y excelente calidad; en otras palabras, se puede depositar grandes cantidades de metal (tres veces más que con el proceso de electrodo revestido) con una buena calidad.
\section*{Meta}

\subsection*{Incorporar una soldadora de micro alambre en un brazo robótico}

\section*{Objetivos}
\subsection{reducción del tiempo al soldar}
\subsection{soldadura más pulcra}
\subsection{ambiente más limpio para el trabajador}




\end{document}